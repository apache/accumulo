
% Licensed to the Apache Software Foundation (ASF) under one or more
% contributor license agreements. See the NOTICE file distributed with
% this work for additional information regarding copyright ownership.
% The ASF licenses this file to You under the Apache License, Version 2.0
% (the "License"); you may not use this file except in compliance with
% the License. You may obtain a copy of the License at
%
%     http://www.apache.org/licenses/LICENSE-2.0
%
% Unless required by applicable law or agreed to in writing, software
% distributed under the License is distributed on an "AS IS" BASIS,
% WITHOUT WARRANTIES OR CONDITIONS OF ANY KIND, either express or implied.
% See the License for the specific language governing permissions and
% limitations under the License.

\chapter{Security}

Accumulo extends the BigTable data model to implement a security mechanism
known as cell-level security. Every key-value pair has its own security label, stored
under the column visibility element of the key, which is used to determine whether
a given user meets the security requirements to read the value. This enables data of
various security levels to be stored within the same row, and users of varying
degrees of access to query the same table, while preserving data confidentiality.

\section{Security Label Expressions}

When mutations are applied, users can specify a security label for each value. This is
done as the Mutation is created by passing a ColumnVisibility object to the put()
method:

\begingroup\fontsize{8pt}{8pt}\selectfont\begin{verbatim}
Text rowID = new Text("row1");
Text colFam = new Text("myColFam");
Text colQual = new Text("myColQual");
ColumnVisibility colVis = new ColumnVisibility("public");
long timestamp = System.currentTimeMillis();

Value value = new Value("myValue");

Mutation mutation = new Mutation(rowID);
mutation.put(colFam, colQual, colVis, timestamp, value);
\end{verbatim}\endgroup

\section{Security Label Expression Syntax}

Security labels consist of a set of user-defined tokens that are required to read the
value the label is associated with. The set of tokens required can be specified using
syntax that supports logical AND \verb^&^ and OR \verb^|^ combinations of terms, as
well as nesting groups \verb^()^ of terms together.

Each term is comprised of one to many alpha-numeric characters, hyphens, underscores or
periods. Optionally, each term may be wrapped in quotation marks
which removes the restriction on valid characters. In quoted terms, quotation marks
and backslash characters can be used as characters in the term by escaping them
with a backslash.

For example, suppose within our organization we want to label our data values with
security labels defined in terms of user roles. We might have tokens such as:

\begingroup\fontsize{8pt}{8pt}\selectfont\begin{verbatim}
admin
audit
system
\end{verbatim}\endgroup

These can be specified alone or combined using logical operators:

\begingroup\fontsize{8pt}{8pt}\selectfont\begin{verbatim}
// Users must have admin privileges:
admin

// Users must have admin and audit privileges
admin&audit

// Users with either admin or audit privileges
admin|audit

// Users must have audit and one or both of admin or system
(admin|system)&audit
\end{verbatim}\endgroup

When both \verb^|^ and \verb^&^ operators are used, parentheses must be used to specify
precedence of the operators.

\section{Authorization}

When clients attempt to read data from Accumulo, any security labels present are
examined against the set of authorizations passed by the client code when the
Scanner or BatchScanner are created. If the authorizations are determined to be
insufficient to satisfy the security label, the value is suppressed from the set of
results sent back to the client.

Authorizations are specified as a comma-separated list of tokens the user possesses:

\begingroup\fontsize{8pt}{8pt}\selectfont\begin{verbatim}
// user possess both admin and system level access
Authorization auths = new Authorization("admin","system");

Scanner s = connector.createScanner("table", auths);
\end{verbatim}\endgroup

\section{User Authorizations}

Each Accumulo user has a set of associated security labels. To manipulate
these in the shell while using the default authorizor, use the setuaths and getauths commands.
These may also be modified for the default authorizor using the java security operations API. 

When a user creates a scanner a set of Authorizations is passed. If the
authorizations passed to the scanner are not a subset of the users
authorizations, then an exception will be thrown.

To prevent users from writing data they can not read, add the visibility
constraint to a table. Use the -evc option in the createtable shell command to
enable this constraint. For existing tables use the following shell command to
enable the visibility constraint. Ensure the constraint number does not
conflict with any existing constraints.
  
\begingroup\fontsize{8pt}{8pt}\selectfont\begin{verbatim}
config -t table -s table.constraint.1=org.apache.accumulo.core.security.VisibilityConstraint
\end{verbatim}\endgroup

Any user with the alter table permission can add or remove this constraint.
This constraint is not applied to bulk imported data, if this a concern then
disable the bulk import permission.

\section{Pluggable Security}

New in 1.5 of Accumulo is a pluggable security mechanism. It can be broken into three actions-
authentication, authorization, and permission handling. By default all of these are handled in
Zookeeper, which is how things were handled in Accumulo 1.4 and before. It is worth noting at this
point, that it is a new feature in 1.5 and may be adjusted in future releases without the standard
deprecation cycle.

Authentication simply handles the ability for a user to verify their integrity. A combination of 
principal and authentication token are used to verify a user is who they say they are. An 
authentication token should be constructed, either directly through it's constructor, but it is 
advised to use the init(Property) method to populate an authentication token. It is expected that a 
user knows what the appropriate token to use for their system is. The default token is 
PasswordToken. 

Once a user is authenticated by the Authenticator, the user has access to the other actions within 
Accumulo. All actions in Accumulo are ACLed, and this ACL check is handled by the Permission 
Handler. This is what manages all of the permissions, which are divided in system and per table 
level. From there, if a user is doing an action which requires authorizations, the Authorizor is 
queried to determine what authorizations the user has.

This setup allows a variety of different mechanisms to be used for handling different aspects of 
Accumulo's security. A system like Kerberos can be used for authentication, then a system like LDAP 
could be used to determine if a user has a specific permission, and then it may default back to the 
default ZookeeperAuthorizor to determine what Authorizations a user is ultimately allowed to use. 
This is a pluggable system so custom components can be created depending on your need.

\section{Secure Authorizations Handling}

For applications serving many users, it is not expected that an Accumulo user
will be created for each application user. In this case an Accumulo user with
all authorizations needed by any of the applications users must be created. To
service queries, the application should create a scanner with the application
user's authorizations. These authorizations could be obtained from a trusted 3rd
party.

Often production systems will integrate with Public-Key Infrastructure (PKI) and
designate client code within the query layer to negotiate with PKI servers in order
to authenticate users and retrieve their authorization tokens (credentials). This
requires users to specify only the information necessary to authenticate themselves
to the system. Once user identity is established, their credentials can be accessed by
the client code and passed to Accumulo outside of the reach of the user.

\section{Query Services Layer}

Since the primary method of interaction with Accumulo is through the Java API,
production environments often call for the implementation of a Query layer. This
can be done using web services in containers such as Apache Tomcat, but is not a
requirement. The Query Services Layer provides a mechanism for providing a
platform on which user facing applications can be built. This allows the application
designers to isolate potentially complex query logic, and enables a convenient point
at which to perform essential security functions.

Several production environments choose to implement authentication at this layer,
where users identifiers are used to retrieve their access credentials which are then
cached within the query layer and presented to Accumulo through the
Authorizations mechanism.

Typically, the query services layer sits between Accumulo and user workstations.
